% !TeX encoding = utf8
% !TeX spellcheck = fr

\chapter{Installation}\label{cha:installation}

Pour lancer XCSoar il vous faut~:
\begin{itemize}
\item un appareil sur lequel faire tourner XCSoar
\item le logiciel XCSoar
\item un récepteur GPS (peut être dans l'appareil)
\item un fichier de points de virage
\item un fichier d'espaces aériens (optionnel)
\item un fichier de terrain, pour la carte (optionnel)
\end{itemize}

\section{!~Avant de faire votre premier vol avec XCSoar~!}

Après avoir installé XCSoar avec succès, vous pourriez l'utiliser directement. XCSoar
démarrera avec une pré-configuration prête à être utilisée. Mais attention qu'à 
ce moment-là votre nouveau jouet ne vous affichera qu'une carte mobile.
\warning \emph{Ne faites pas confiance aux calculs.} Il faut dire à l'avance à XCsoar sur quel
aéronef vous volez. Ceci est fait en précisant les données sur votre aéronef
telles que la polaire, sa masse et d'autres données. Cependant, il est toujours bon
d'étudier le manuel et de se familiariser avec XCSoar à la maison.

\section{Comment tirer le maximum d'XCSoar}

Afin de bénéficier au maximum d'XCSoar, il vous est demandé de
faire des choses supplémentaires après l'installation du logiciel et de télécharger quelques fichiers
de données. ``Des choses supplémentaires'' incluent des données personnelles et sur l'aéronef, ainsi que la configuration
de quelques fonctionnalités. A moins que vous souhaitiez peaufiner tout ce que XCSoar
fournit, cela peut se faire assez rapidement. Les étapes nécessaires sont
résumées dans une \emph{Checklist d'XCSoar}, fournit dans le paragraphe suivant.

Si vous comptez utiliser un appareil avec plusieurs composants connectés,
ce manuel vous donnera des conseils utiles à la fois sur la configuration des choses et sur
leur utilisation.

Si vous êtes un pilote pressé, les auteurs vous suggère d'aller voir la version courte
du manuel \texttt{XCSoar-Prise-en-main} et de parcourir la \emph{Checklist d'XCSoar}
étape par étape. La version courte du manuel est disponible sur \url{http://www.xcsoar.org/discover/manual.html}.

\section{Checklist d'XCSoar}

\subsection*{{Faire entrer XCSoar dans la danse}}
\begin{itemize}
\item acquérir un appareil et y installer XCSoar
\item récupérer les fichiers de données adapté à votre secteur de vol
\item dire à XCSoar quels fichiers utiliser
\item dire à XCSoar la polaire et la masse de votre planeur
\item éventuellement, se connecter aux instruments
\item terminer la configuration et se familiariser
\item monter l'appareil
\item ajouter les points listés ci-dessous dans vos checklists
\item en faire un outil usuel
\end{itemize}

\subsection*{Faire les vérifications pré-vols, incluant~:}
\begin{itemize}
\item configurer la polaire et la masse
\item configurer le vent et les paramètres de vol (MC, moucherons, QNH)
\item éventuellement, configurer un circuit.
\end{itemize}

\subsection*{Faire les vérifications avant le départ, incluant~:}
\begin{itemize}
\item vérifier le vent et la configuration de vol une fois de plus
\end{itemize}
\vspace{2em}

\subsection*{Voler, prendre du plaisir}
\vspace{4em}

\subsection*{Faire les vérification post-vol}
\begin{itemize}
\item Télécharger les logs du vol depuis l'enregistreur, les mettre sur skylines, la netcoupe et l'OLC
\item Récupérer les données statistiques sur le vol.
\end{itemize}
\newpage

\section{Compatibilité}

\subsection*{Appareils faisant tourner XCSoar}

XCSoar tourne sur les plateformes suivantes~:
\begin{itemize}
\item téléphones portables et tablettes sous Android 1.6 ou plus récent \\
 Exemples~: Dell Streak, Samsung Galaxy S II, HTC Desire HD,
 Motorola Xoom
\item eReader Kobo
\item Windows 2000 ou plus récent
\item Linux
\item Mac OS X (obsolète)
\end{itemize}

\subsection*{GPS, Enregistreur de Vol, Vario}

XCSoar est compatible avec tous les GPS fournissant des données NMEA. La plupart
des appareils Android intègre un GPS, mais, pour diverses raisons, il est désirable de
se connecter à un ou plusieurs appareils externes~:
\begin{itemize}
\item un récepteur GPS spécialisé possède une bien meilleure réception, fournissant de bien
meilleures données pour les mesures et les calculs
\item un indicateur de vitesse fournit de manière exacte et rapide des estimations du vent
 sans nécessiter de spiraler
\item un variomètre améliore l'assistant en thermique
\item un circuit peut être déclaré dans un enregistreur de vol IGC et, après l'atterrissage, l'enregistrement
du vol peut être téléchargé
\item certains variomètres permettent la synchronisation de la configuration du MacCready avec
 XCSoar
\item un FLARM (voire une entrée ADS-B) fournit des informations et des états sur les autres
autour de vous (et bien sûr, un FLARM détecte des risques de collision)
\end{itemize}

\subsection*{Appareils externes compatibles et fonctionnalités}
\label{sec:supported-varios}

\newcommand{\y}[0]{{ $\surd$ }}
%{0.8\textwidth}
\noindent\makebox[\textwidth]{%
\begin{tabular}{l|ccc|cc|cc|c}
    \multicolumn{1}{c}{Compatibles} & \multicolumn{3}{c|}{Fonctionnalités} & \multicolumn{5}{c}{Flux de données} \\
NMEA Device & 
 \begin{sideways} Déclaration\end{sideways} & 
 \begin{sideways} Télécommande\end{sideways} & 
 \begin{sideways} Téléchargement\end{sideways} &
 \begin{sideways} Vitesse air\end{sideways} & 
 \begin{sideways} Vario\end{sideways} & 
 \begin{sideways} Baro.\ altitude\end{sideways} &
 \begin{sideways} Vent\end{sideways} &
 \begin{sideways} Accéléromètre\end{sideways} \\
\hline
%                    _Decl_Remo_Down_Airs_Vari_Baro_Wind_Gsen_
Borgelt B50          &    & \y &    & \y & \y & \y &    &    \\
CAI 302              & \y & \y & \y & \y & \y & \y & \y & \y \\
CAI GPS Nav          &    &    &    &    &    &    &    &    \\
Condor               &    &    &    & \y & \y & \y & \y &    \\
\hline
Digifly Leonardo     &    &    &    & \y & \y & \y & \y &    \\
EW Logger            & \y &    &    &    &    & \y &    &    \\
EW microRecorder     & \y &    &    &    &    & \y &    &    \\
FLARM                & \y &   & \y  &    &    & \y &    &    \\
\hline
%                    _Decl_Remo_Down_Airs_Vari_Baro_Wind_Gsen_
Flymaster F1         &    &    &    &    & \y & \y &    &    \\
Flytec 5030          &    &    &    & \y & \y &    &    &    \\
GTAltimeter          &    &    &    &    &(\y)& \y &    &    \\
ILEC SN10            &    &    &    &    & \y & \y & \y &    \\
\hline
IMI ERIXX            & \y &    & \y &    &    &    &    &    \\
LX20, Colibri        & \y &    & \y &    &    & \y &    &    \\
LXNAV Nano           & \y &    & \y &    &    &    &    &    \\
\hline
%                    _Decl_Remo_Down_Airs_Vari_Baro_Wind_Gsen_
LXNAV V7             &    & \y &    & \y & \y &    &    &    \\
PosiGraph            & \y &    &    &    &    & \y &    &    \\
Triadis Altair (pro) & \y &    &    &    &    & \y &    &    \\
Triadis Vega         &    & \y &    & \y & \y & \y &    & \y \\
\hline
Vaulter              &    & \y &    & \y & \y & \y & \y & \y \\
Volkslogger          & \y &    & \y &    &    & \y &    &    \\
Westerboer VW1150    &    & \y &    & \y & \y & \y &    &    \\
Westerboer VW921, 922
                     &    & \y &    & \y & \y & \y &    &    \\
\hline
Zander / SDI         &    & \y &    & \y & \y & \y & \y &    \\

\end{tabular}}

Alors que la majorité des appareils sous Windows CD ont un port série, des matériels
aussi anciens ne sont pas présent sur les appareils Android modernes. Ces derniers peuvent soit
utiliser le Bluetooth ou une carte Android IOIO. Pour utiliser le Bluetooth, vous devez
connecter le périphérique externes à un adaptateur Bluetooth-vers-Série, tels
que le K6-Bt et le SoarTronic-BT1/2.

\section{Installation du logiciel}

Le logiciel est téléchargeable gratuitement sur le site internet d'XCSoar ~\xcsoarwebsite{}. Ce paragraphe décrit quel fichier doit être téléchargé, et comment l'installer.

\subsection*{Sous Android}

Récupérez XCSoar sur Google Play depuis votre appareil, ou installez le fichier \verb|apk|
manuellement. Copiez les fichiers de données sur la carte SD dans le répertoire \verb|XCSoarData|.

\subsection*{Sur un Kobo Mini}

Le Kobo Mini est un lecteur d'e-book pas cher. Il a un affichage
e-paper noir et blanc qui possède une excellente lisibilité en plein soleil.

Avant de commencer, faites une sauvegarde de la carte SD interne. L'installateur d'XCSoar
pourrait casser votre Kobo, quoi que ce soit improbable. Vous pouvez toujours récupérer le Kobo
d'une défaillance logicielle, mais uniquement si vous avez accès à une sauvegarde.

Pour installer XCSoar, connecter par USB le Kobo à votre PC. Le Kobo 
apparaît comme un périphérique de stockage sur votre PC~; ouvrez-le et créer un
répertoire nommé \texttt{.kobo} (noter le point initial). Téléchargez le 
fichier \texttt{KoboRoot.tgz} depuis le site internet d'XCSoar vers ce
répertoire (\url{http://www.xcsoar.org/hardware/}). Débrancher le Kobo et le redémarrer (arrêter le complètement
puis allumer le à nouveau). Vous verrez le message ``Updating'' (``mise à jour'') et
après quelques minutes, le Kobo affichera un menu qui vous permet de lancer
XCSoar ou le logiciel de lecteur d'\mbox{e-book} du Kobo.

Pour copier les fichiers de données (cartes, points de virage...) sur le Kobo, lancer le 
logiciel original du Kobo (``Nickel'') et connecter de nouveau le Kobo à votre PC.
Copier les fichiers dans un répertoire nommé \texttt{XCSoarData} à la racine.

Autrement, des fichiers de données peuvent être téléchargés depuis le gestionnaire de fichiers d'XCSoar,
en ayant démarré une connexion réseau avant le lancement d'XCSoar.

\subsubsection{Bidouiller le Kobo}

Suite à l'installation d'XCSoar sur le Kobo, le nouveau script de démarrage vérifie qu'un
script nommé \texttt{XCSoarData/kobo/init.sh} existe et le lance. Si
vous vous y connaissez, vous pouvez utiliser ce script pour effectuer des choses au
moment du démarrage, comme lancer \texttt{inetd} (pour un accès \texttt{telnet}).

Quand vous lancer \texttt{Nickel} (le système original de l'e-book), le nouveau
script de démarrage cherche aussi un script nommé \texttt{init\_nickel.sh}
dans \texttt{XCSoarData/kobo/} et le lance. Là aussi, si 
vous vous y connaissez vous pouvez utiliser ce script pour effectuer des choses
avant que \texttt{Nickel} n'est complètement démarré, comme envoyer des instructions
à votre vario apparié (pour l'éteindre, pour baisser le volume, etc.).

\subsection*{Sur un PC sous Windows}

Téléchargez le fichier programme \verb|XCSoar.exe| (cible ``PC'') sur votre disque dur.

\subsection*{Sous Unix/Linux}

Téléchargez \verb|xcsoar_XXX.deb|, où \verb|XXX| contient les numéros de version et la plateforme, par~ex.\ \verb|xcsoar_6.0.4_i386.deb|.
Il s'agit d'un paquet Debian et peut être installé ainsi 
\begin{center}
\verb|sudo dpkg -i xcsoar_XXX.deb|.
\end{center}
Utilisez \verb|dpkg-query -L xcsoar| pour voir où l'exécutable et les autres fichiers sont installés.
Les fichiers additionnels doivent être placés dans le répertoire
\verb|~/.xcsoar/|.
Si \verb|~/.xcsoar| n'existe pas, il sera créé la première fois que \verb|xcsoar| sera exécuté.

\subsection*{Sur une Raspberry Pi et une Cubieboard}

Installer le paquetage Debian tel que décrit ci-dessous. Toutefois, contrairement à 
Linux ``normal'', XCSoar n'utilisera pas X11. À la place, il tournera
en mode plein écran dans une console Linux.

XCSoar a besoin d'accéder à vos périphériques d'entrée
(\texttt{/dev/input/event*}). Par défaut, seul \texttt{root} y
a accès. Pour passer outre, créez un fichier de configuration
\texttt{udev}, par ex. \texttt{/etc/udev/rules.d/99-input.rules}~:

\begin{verbatim*}
KERNEL=="event*", NAME="input/%k", MODE="660", GROUP="input"
\end{verbatim*}

Créez le groupe \texttt{input} et en rendre membre votre utilisateur~:

\begin{verbatim*}
groupadd input
adduser pi input
\end{verbatim*}

\section{Fichiers de données}\label{sec:data files}

Pour être capable d'utiliser les fonctionnalités avancées d'XCsoar, des fichiers de données supplémentaires, tels que
le relief, la topographie, des espaces aériens à usage particulier, des points de virage, etc.\, sont nécessaires. Les fichiers
qui peuvent être utilisés avec XCSoar sont décrits dans le chapitre~\ref{cha:data-files}.

Tous les fichiers de données doivent être copiés dans le répertoire 
\texttt{XCSoarData}. Ce répertoire doit être dans un endroit spécifique
de telle façon qu'XCSoar sache où rechercher des fichiers de données~:
\begin{description}
\item[PC sous Windows]
\texttt{XCSoarData} est dans votre répertoire personnel (``\texttt{Mes
Documents}'')
\item[Windows Mobile PDA/PNA]
S'il y a un répertoire nommé \texttt{XCSoarData} dans le même
répertoire que \texttt{XCSoar.exe}, alors celui-ci sera utilisé.
\texttt{XCSoarData} est sur une carte SD. Si il n'y a pas de carte SD, alors
XCSoar le cherchera dans \texttt{Mes Documents}.
\item[Unix/Linux]
Le répertoire s'appelle \verb|.xcsoar| dans le répertoire principal de l'utilisateur.
\item[Appareil Android]
\texttt{XCSoarData} est sur la carte SD.
\end{description}

XCSoar générera un certain nombre de fichiers supplémentaires lors de l'exécution. Ils
seront placés dans le répertoire \texttt{XCSoarData} (PS sous Windows, 
appareils mobiles sous Windows et Android), ou dans le répertoire \texttt{.xcsoar} (PC sous
Unix/Linux). Au moment du premier démarrage, XCSoar va créer et mettre à jour les fichiers
\texttt{user.cup} (points de virage édités par l'utilisateur),
\texttt{Default.tsk} (circuit par défaut), 
\texttt{default.prf} 
(paramètres de configuration),
\texttt{xcsoar.log}, 
ainsi que trois répertoires~: \texttt{cache},
\texttt{config} et \texttt{logs}. Des fichiers supplémentaires peuvent être 
créés/modifiés pendant que XCSoar tourne, tels que des fichiers de circuit
(\texttt{*.tsk}) et les enregistrements du vol.


\section{Faire tourner XCSoar}
%\subsection*{Fly and simulator modes}

Le logiciel XCSoar peut fonctionner selon deux modes~:
\begin{description}
\item[VOL] Ce mode est utilisé lorsque vous êtes vraiment en vol. Le simulateur est
 désactivé et les communications séries sont actives. 
\item[SIM] Cela lance XCSoar en mode simulateur, sans tenter de démarrer
 les communications séries.
\end{description}

\subsection*{Version PC de XCSoar}
Le logiciel peut être lancé en ouvrant la fenêtre de l'explorateur, en trouvant le répertoire
qui contient l'exécutable XCSoar.exe, et en double-cliquant sur le fichier du logiciel.

Les options en ligne de commande du logiciel permettent de définir
l'orientation de l'affichage~:
\begin{description}
\item[-portrait] L'écran fait 480~pixels de large, 640~pixels de haut.
\item[-square] L'écran fait 480~pixels de large, 480~pixels de haut.
\item[-landscape] L'écran fait 640~pixels de large, 480~pixels de haut. C'est la 
configuration par défaut. Si vous ne spécifier ce paramètre, la version paysage sera
chargé automatiquement.
\item[-small] Diminue la taille de l'écran de moitié. C'est utile pour se servir d'XCSoar en
même temps que de simulateurs de vol comme Condor.
\end{description}
Pour changer l'orientation de l'écran, il est pratique de créer un raccourci vers le
logiciel et, en faisant un clic droit sur l'icône du raccourci puis sur ``Propriétés''.
Dans le champ ``Cible'' ajouter l'option désirées parmi celles listées ci-dessus.

\subsection*{Version Unix/Linux de XCSoar}
Lancer \verb|xcsoar| à partir de la ligne de commande, ou créer un raccourci sur le 
bureau. La localisation du fichier exécutable peut être trouvé en utilisant
\verb|which xcsoar|. Pour le moment, seul le mode paysage est disponible.

\subsection*{Charger les fichiers de données}\label{sec:loaddatafiles}
La première fois qu'XCSoar est lancé, il ne charge pas automatiquement les
fichiers de données que vous avez placés dans le répertoire \verb|XCSoarData|. 
Pour dire à XCSoar quels fichiers charger double cliquer/tapoter la carte (la grande
surface blanche avec un symbole de planeur au centre),
Choisissez le menu \bmenug{Config 2} (cliquer/tapoter-le deux fois), puis sélectionne l'élément
\mbox{\bmenug{Système}.} L'écran de configuration devrait s'afficher~:
\sketch{figures/config-basic.png}
La première page vous permet de choisir les fichiers de carte,
de points de virage et d'espace aérien en cliquant/tapotant sur les zones de texte.
De nombreuses autres fonctionnalités d'XCSoar peuvent être ici configurées. Elles sont décrites en détails dans le
chapitre~\ref{cha:configuration}.
Une fois fait, XCSoar rechargera ces fichiers~: à partir de maintenant les fichiers de données
seront chargés automatiquement au démarrage. 

\subsection*{Démarrage et profils utilisateur}\label{sec:profiles}
Quand XCSoar démarre, il vérifiera les profils existants. Si plusieurs
profils sont détectés il affichera une petite fenêtre vous demandant quel profil
charger. Pour continuer, choisissez le profil désiré et appuyés sur Entrée. Si aucun
profil n'est choisi les paramètres de la session précédente seront chargés à nouveau. Des profils
peuvent être utiles dans les cas suivants~:
\begin{itemize}
\item différents pilotes
\item compétition ou vol de d'agrément
\item voler depuis différents endroits
\item différents aéronefs (avec différentes polaires)
\end{itemize}
Des profils peuvent aussi servir à conserver la sauvegarde d'une
configuration particulière. Presque tous les paramètres sont enregistrés dans un fichier
de profil avec l'extension \texttt{.prf}. Une fois que vous êtes satisfait de votre paramétrage,
faites deux copies de votre fichier de profil. Une première ayant l'extension \texttt{.prf}, 
une seconde avec l'extension \texttt{.bak}.
Alors que le fichier \texttt{.prf} se lancera au démarrage et contiendra tous 
vos changements depuis que XCSoar tourne jusqu'au prochain démarrage, le fichier
\texttt{.bak} préservera vos paramètres, tant que vous jugerez utile de le faire.
A titre d'exemple, vous pourriez créer une série de fichiers tels que~:
\begin{itemize}
\item \texttt{CopainsEnArcus.prf}
\item \texttt{CopainsEnArcus.bak}
\item \texttt{JohnEnKa6auPaysDesMerveilles.prf}
\item \texttt{JohnEnKa6auPaysDesMerveilles.bak}
\end{itemize}

\subsection*{Mode SIM}
XCSoar possède une interface simple permettant de faire un vol
simulé. Selon l'appareil utilisé, il y a différentes
méthodes pour modifier les valeurs de cap, de vitesse et de hauteur. La simulation est
prévue pour une première familiarisation avec XCSoar en action. Si vous appréciez
l'idée d'une simulation plus réaliste à la maison, vous devriez acquérir un ``vrai''
simulateur de vol plané, à connecter à XCSoar.

Sur l'écran de la carte, cliquer/toucher le symbole du planeur
et le déplacer crée un déplacement du planeur dans cette direction, la
vitesse étant proportionnelle à la longueur du déplacement.

Avec des boutons, la vitesse, l'altitude et le cap de l'aéronef
peuvent être changés en utilisant les Infoboxes.
Ce qui suit pourrait ne pas être complètement disponible sur tous les appareils, mais sur
tout appareil faisant tourner XCSoar, toutes les entrées nécessaires pour
réaliser une simulation existent.

En appuyant sur une Infoboxe vous sélectionnez une valeur à modifier avec des boutons
réels ou du menu.
L'altitude de l'aéronef peut être modifiées en sélectionnant l'Infoboxe
d'altitude GPS \bmenuw{Alt GPS}, puis en utilisant les touches haut et bas ou les boutons
de l'écran tactile.
La vitesse par rapport à l'air peut être modifiée en sélectionnant l'Infoboxe de vitesse sol
\bmenuw{V Sol}, puis en utilisant les touches haut et bas ou les boutons du menu.
La trace du planeur peut être modifiée en sélectionnant l'InfoBoxe de trace 
\bmenuw{Trace}, puis en utilisant les touches haut et bas ou les boutons du menu.

En ayant sélectionné soit \bmenuw{Alt GPS} soit \bmenuw{V Sol},
le cap du planeur peut être modifié en utilisant les touches gauche/droite.

Les autres commandes, boutons et menus fonctionnent de la même façon qu'en mode VOL.


\subsection*{Écran d'accueil}
Quand XCSoar démarre, s'éteint ou charge de gros fichiers tels que les espaces aériens,
les points de virage, le relief, etc., un écran de progression est affiché pendant que les données sont
en train d'être chargée. Cet écran a une barre de progression qui indique l'activité
de chargement des données, et une courte ligne de texte décrivant l'action en cours.

Cet écran affiche aussi les information sur la version du logiciel.

\subsection*{Quitter le logiciel}
Sur les versions PDA ou PC, XCSoar est éteint en utilisant le menu. Ce menu peut être
ouvert en double-cliquant sur la carte ou sur l'\InfoBox 
\begin{quote}
\bmenug{QUIT}
\end{quote}

Sur les versions PC, XCSoar peut aussi être fermé en cliquant sur l'icône de fermeture
de la fenêtre XCSoar.
