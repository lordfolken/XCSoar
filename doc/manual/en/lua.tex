\lstnewenvironment{lua}
{\lstset{language=[5.0]Lua}}
{}

Starting with version 7.0, \xc can be extended using
\href{http://www.lua.org/}{Lua} scripts.

Lua is a language that is easy to learn, powerful enough for XCSoar
and light: the interpreter library weighs just 200 kB.  Lua is a
common language choice for integrated scripting languages.

\section{Learning Lua}

\begin{lua}
print("Hello World")
\end{lua}

This manual will not attempt to teach you basic Lua.  There are enough
resources on the internet, for example:

\begin{itemize}
\item \href{http://www.lua.org/manual/5.3/}{Lua 5.3 Reference Manual}
\item \href{http://www.lua.org/pil/contents.html}{Programming in
  Lua}, a book on Lua
\item \href{http://lua-users.org/wiki/TutorialDirectory}{Tutorial
  Directory} on the lua-users wiki
\item \href{https://en.wikipedia.org/wiki/Lua_%28programming_language%29}{Wikipedia}
\end{itemize}

Just to get you started from here, here's some more example code:

\begin{lua}
-- comment starts with a double hyphen

--[[
multi
line
comment
]]--

i = 42
if i > 1 then
   print("i=" .. i)
elseif i == 0 then
   print("zero")
else
   error("negative")
end

a = {1, 'a', 3.14}
print(a[2])

function f(a, b)
   return a * b
end
print(f(2, 3))
\end{lua}

\section{Running Lua}

The directory \texttt{XCSoarData/lua/} may contain Lua scripts
(\texttt{*.lua}).  The directory \texttt{XCSoarData/lua/lib/} may
contain Lua libraries to be loaded with \texttt{require}.

After startup, \xc starts the script \texttt{init.lua} (if it
exists).

The \emph{InputEvent} ``\texttt{RunLuaFile}'' can be used to start
additional scripts.  If no parameter is given, the user is asked to
choose a file.  Note that the \emph{InputEvent} subsystem is
deprecated and will be removed once Lua support is complete.

As long as a Lua script runs, the \xc user interface is blocked.
Be careful not to write scripts that loop forever.

Once the Lua script finishes, the Lua interpreter is shut down --
unless the script has registered a callback (e.g. a \verb|timer|).  In
that case, the Lua script stays resident until it unregisters all
callbacks (or until \xc quits or the user stops the script
explicitly).

\section{Lua Standard Libraries}

\xc enables the following Lua standard libraries:

\begin{itemize}
\item \verb|package|
\item \verb|table|
\item \verb|string|
\item \verb|math|
\end{itemize}

Lua's \verb|print()| function writes to the \xc log file
(\texttt{XCSoarData/xcsoar.log}).

The \verb|error()| function aborts the Lua script and reports the
specified error message to the user.

\xc adds another function to the root namespace: \verb|alert()|.
It shows a dialog with the specified message, and returns as soon as
the user has closed the dialog.  This function is experimental, and
may disappear or be renamed at any time.  Most importantly: do not
abuse it, as it may annoy the user.

\section{\xc's Lua API}

The package/namespace \verb|xcsoar| provides access to \xc.  It
contains the following names:

\begin{maxipage}
\begin{tabular}{l|p{13cm}}
  Name & Description \\
  \hline
  \hline

  \verb|VERSION| & The \xc version number, for example
  ``\texttt{7.0}''. \\
  \hline
  \verb|blackboard| & Access to sensor data. \\
  \hline
  \verb|map| & The map view. \\
  \hline
  \verb|timer| & Class for scheduling periodic callbacks. \\
\end{tabular}
\end{maxipage}

\subsection{The Blackboard}

The blackboard provides access to sensor data, such as GPS location.

The following attributes are provided by \verb|xcsoar.blackboard|:

\begin{maxipage}
\begin{tabular}{l|p{12cm}}
Name & Description \\
\hline\hline

\verb|location| & The current location (table with
keys \verb|longitude| and \verb|latitude| in degrees) according to
GPS. \\

\hline

\verb|altitude| & The current altitude $[m]$ above MSL. \\

\hline

\verb|track| & The current flying direction above ground in degrees. \\

\hline

\verb|ground_speed| & The aircraft speed relative to the ground
$[{m \over s}]$. \\

\hline

\verb|air_speed| & The true airspeed
$[{m \over s}]$. \\

\hline

\verb|bank_angle| & The bank angle in degrees. \\

\hline

\verb|pitch_angle| & The pitch angle in degrees. \\

\end{tabular}
\end{maxipage}

Any of these may be \verb|nil| if its value is not known, e.g. if
there is no GPS fix.

\subsection{The Map}

The map provides access to XCSoar's map view.

The following attributes are provided by \verb|xcsoar.map|:

\begin{maxipage}
\begin{tabular}{l|p{12cm}}
Name & Description \\
\hline\hline

\verb|location| & The current reference location (may be aircraft location or
pan location). \\

\hline

\verb|show()| & Show the map; disable thermal assistant or other
widgets replacing the map view. \\

\end{tabular}
\end{maxipage}

\subsection{Timers}

The class \verb|xcsoar.timer| implements a timer that calls a given
Lua function periodically.

\begin{lua}
xcsoar.timer.new(60, function(t)
  print("A minute has passed")
  t:cancel()
end)
\end{lua}

The following methods are available in \verb|xcsoar.timer|:

\begin{maxipage}
\begin{tabular}{l|p{10cm}}
Method & Description \\
\hline\hline

\verb|new(period, function)| & Create a new instance and schedule
it.  The period is a numeric value in seconds. \\

\hline

\verb|cancel()| & Cancel the timer. \\

\hline

\verb|schedule(period)| & Reschedule the timer. \\

\end{tabular}
\end{maxipage}

\subsection{Legacy}

Before version 7.0, \xc was adapted using the \emph{InputEvent}
subsystem (see Appendix~\ref{cha:file_formats}).  During the Lua
transition, Lua scripts can invoke InputEvents, for example:

\begin{lua}
xcsoar.fire_legacy_event("Setup", "basic")
xcsoar.fire_legacy_event("Zoom", "basic")
\end{lua}

This function will be removed before the final 7.0 release.
