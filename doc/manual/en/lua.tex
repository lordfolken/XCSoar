\lstnewenvironment{lua}
{\lstset{language=[5.0]Lua}}
{}

Starting with version 7.0, \xc can be extended using
\href{http://www.lua.org/}{Lua} scripts.

Lua is a language that is easy to learn, powerful enough for XCSoar
and light: the interpreter library weighs just 200 kB.  Lua is a
common language choice for integrated scripting languages.

\section{Learning Lua}

\begin{lua}
print("Hello World")
\end{lua}

This manual will not attempt to teach you basic Lua.  There are enough
resources on the internet, for example:

\begin{itemize}
\item \href{http://www.lua.org/manual/5.3/}{Lua 5.3 Reference Manual}
\item \href{http://www.lua.org/pil/contents.html}{Programming in
  Lua}, a book on Lua
\item \href{http://lua-users.org/wiki/TutorialDirectory}{Tutorial
  Directory} on the lua-users wiki
\item \href{https://en.wikipedia.org/wiki/Lua_%28programming_language%29}{Wikipedia}
\end{itemize}

Just to get you started from here, here's some more example code:

\begin{lua}
-- comment starts with a double hyphen

--[[
multi
line
comment
]]--

i = 42
if i > 1 then
   print("i=" .. i)
elseif i == 0 then
   print("zero")
else
   error("negative")
end

a = {1, 'a', 3.14}
print(a[2])

function f(a, b)
   return a * b
end
print(f(2, 3))
\end{lua}

\section{Running Lua}

The directory \texttt{XCSoarData/lua/} may contain Lua scripts
(\texttt{*.lua}).  The directory \texttt{XCSoarData/lua/lib/} may
contain Lua libraries to be loaded with \texttt{require}.

After startup, \xc starts the script \texttt{init.lua} (if it
exists).

The \emph{InputEvent} ``\texttt{RunLuaFile}'' can be used to start
additional scripts.  If no parameter is given, the user is asked to
choose a file.  Note that the \emph{InputEvent} subsystem is
deprecated and will be removed once Lua support is complete.

As long as a Lua script runs, the \xc user interface is blocked.
Be careful not to write scripts that loop forever.

Once the Lua script finishes, the Lua interpreter is shut down --
unless the script has registered a callback (e.g. a \verb|timer|).  In
that case, the Lua script stays resident until it unregisters all
callbacks (or until \xc quits or the user stops the script
explicitly).

\section{Lua Standard Libraries}

\xc enables the following Lua standard libraries:

\begin{itemize}
\item \verb|package|
\item \verb|table|
\item \verb|string|
\item \verb|math|
\end{itemize}

Lua's \verb|print()| function writes to the \xc log file
(\texttt{XCSoarData/xcsoar.log}).

The \verb|error()| function aborts the Lua script and reports the
specified error message to the user.

\xc adds another function to the root namespace: \verb|alert()|.
It shows a dialog with the specified message, and returns as soon as
the user has closed the dialog.  This function is experimental, and
may disappear or be renamed at any time.  Most importantly: do not
abuse it, as it may annoy the user.

\section{\xc's Lua API}

The package/namespace \verb|xcsoar| provides access to \xc.  It
contains the following names:

\begin{maxipage}
\begin{tabularx}{1.9\textwidth}{l|X}
  Name & Description \\
  \hline
  \hline

  \verb|VERSION| & The \xc version number, for example
  ``\texttt{7.0}''. \\
  \hline
  \verb|blackboard| & Access to sensor data. \\
  \hline
  \verb|map| & The map view. \\
  \hline
  \verb|airspace| & Access to airspace data. \\
  \hline
  \verb|wind| & Access to wind data and settings. \\
  \hline
  \verb|timer| & Class for scheduling periodic callbacks. \\
\end{tabularx}
\end{maxipage}

\subsection{The Blackboard}

The blackboard provides access to sensor data, such as GPS location.

The following attributes are provided by \verb|xcsoar.blackboard|:

\begin{maxipage}
\begin{tabularx}{1.9\textwidth}{l|X}
Name & Description \\
\hline\hline

\verb|location| & The current location (table with
keys \verb|longitude| and \verb|latitude| in degrees) according to
GPS. \\

\hline

\verb|altitude| & The current altitude $[m]$ above MSL. \\

\hline

\verb|track| & The current flying direction above ground in degrees. \\

\hline

\verb|ground_speed| & The aircraft speed relative to the ground
$[{m \over s}]$. \\

\hline

\verb|air_speed| & The true airspeed
$[{m \over s}]$. \\

\hline

\verb|bank_angle| & The bank angle in degrees. \\

\hline

\verb|pitch_angle| & The pitch angle in degrees. \\

\hline

\verb|heading| & The current magnetic heading in degrees. \\

\hline

\verb|g_load| & The current g-load. \\

\hline

\verb|static_pressure| & The static pressure
$[{Pa}]$. \\

\hline

\verb|pitot_pressure| & The pitot pressure
$[{Pa}]$. \\

\hline

\verb|dynamic_pressure| & The dynamic pressure
$[{Pa}]$. \\

\hline

\verb|temperature| & The current temperature. \\

\hline

\verb|humidity| & The current humidity \\

\hline

\verb|voltage| & The external battery voltage
$[{V}]$. \\

\hline

\verb|battery_level| & The internal battery-level in percent. \\

\hline

\verb|noncomp_vario| & The non-compensated vertical speed
$[{m \over s}]$. \\

\hline

\verb|total_energy_vario| & The total-energy-compensated vertical speed
$[{m \over s}]$. \\

\hline

\verb|netto_vario| & The netto variometer value
$[{m \over s}]$. \\

\end{tabularx}
\end{maxipage}

Any of these may be \verb|nil| if its value is not known, e.g. if
there is no GPS fix.

\subsection{The Map}

The map provides access to XCSoar's map view.

The following attributes are provided by \verb|xcsoar.map|:

\begin{maxipage}
\begin{tabularx}{1.9\textwidth}{l|X}
Name & Description \\
\hline\hline

\verb|location| & The current reference location (may be aircraft location or
pan location). \\

\hline

\verb|is_panning| & Gives back if the panning mode is active at the moment.\\

\hline

\verb|enterpan()| & Activates the panning mode.\\

\hline

\verb|disablepan()| & Disables the panning mode.\\

\hline

\verb|leavepan()| & Leaves the panning mode.\\

\hline

\verb|panto(float latitude, float longitude)| & Pans to the given location.\\

\hline

\verb|pancursor(int dx, int dy)| & Pans the cursor by dx and dy. \\

\hline

\verb|show()| & Show the map; disable thermal assistant or other
widgets replacing the map view. \\

\end{tabularx}
\end{maxipage}

\subsection{Airspace}

The Airspace provides access to airspace data, such as name / distance
to the next airspace.

The following attributes are provided by \verb|xcsoar.airspace|:

\begin{maxipage}
\begin{tabularx}{1.9\textwidth}{l|X}
Name & Description \\
\hline\hline

\verb|nearest_vertical_distance| & The vertical distance to the next airspace
$[{m}]$.\\

\hline

\verb|nearest_vertical_name| & The name of the next vertical airspace. \\

\hline

\verb|nearest_horizontal_distance| & The horizontal distance to the next airspace 
$[{m}]$. \\

\hline

\verb|nearest_horizontal_name| & The name of the next horizontal airspace. \\

\end{tabularx}
\end{maxipage}

\subsection{Task}

The Task provides access to task data such as distances / bearing to the 
next waypoint.

The following attributes are provided by \verb|xcsoar.task|:

\begin{maxipage}
\begin{tabularx}{1.9\textwidth}{l|X}
Name & Description \\
\hline\hline

\verb|bearing| & The true bearing to the next waypoint. 
For AAT tasks, this is the true \newline bearing 
to the target within the AAT sector. $[{degrees}]$\\

\hline

\verb|bearing_diff| & The difference between the glider's track bearing, 
to the bearing of \newline the next waypoint, or for AAT tasks, to the bearing 
to the target within \newline the AAT sector $[{degrees}]$.\\

\hline

\verb|radial| & The true bearing from the next waypoint 
to your position. $[{degrees}]$. \\

\hline

\verb|next_distance| & The distance to the currently selected waypoint. 
For AAT tasks, this \newline is the distance to the target within the AAT sector.
$[{m}]$ \\

\hline

\verb|next_distance_nominal| & The distance to the currently selected waypoint. 
For AAT tasks, this \newline is the distance to the origin of the AAT sector.
$[{m}]$ \\

\hline

\verb|next_ete| &  Estimated time required to reach next waypoint, 
assuming \newline performance of ideal MacCready cruise/climb cycle.\\

\hline

\verb|next_eta| & Estimated arrival local time at next waypoint, 
assuming performance \newline of ideal MacCready cruise/climb cycle. \\

\hline

\verb|next_altitude_diff| & Arrival altitude at the next waypoint relative 
to the safety arrival height. \\

\hline

\verb|nextmc0_altitude_diff| & Arrival altitude at the next waypoint with 
MC 0 setting, relative to the \newline safety arrival height. \\

\hline

\verb|next_altitude_require| & Additional altitude required to reach the next
turnpoint. \\

\hline

\verb|next_altitude_arrival| & Absolute arrival height at the next waypoint in 
final glide. \\

\hline

\verb|next_gr| & The required glide ratio over ground to reach the next waypoint, 
\newline given by the distance to the next waypoint divided by the height 
\newline required to arrive at the safety arrival height. \\

\hline

\verb|final_distance| & Distance to finish around remaining turn points. \\

\hline

\verb|final_ete| & Estimated time required to complete task, 
assuming performance \newline of ideal MacCready cruise/climb cycle. \\

\hline

\verb|final_eta| & Estimated arrival local time at task completion, 
assuming performance \newline of ideal MacCready cruise/climb cycle. \\

\hline

\verb|final_altitude_diff| & Arrival altitude at the final task turn point
relative to the safety arrival \newline height. \\

\hline

\verb|finalmc0_altitude_diff| & Arrival altitude at the final task turn point
, with MC 0 setting, relative \newline to the safety arrival height. \\

\hline

\verb|final_altitude_require| & Additional altitude required to finish 
the task. \\

\hline

\verb|task_speed| & Average cross country speed while on the current 
task, \newline not compensated for altitude. \\

\hline

\verb|task_speed_achieved| & Achieved cross country speed while on the 
current task, \newline compensated for altitude. Equivalent to 
Pirker cross country \newline speed remaining. \\

\hline

\verb|task_speed_instant| & Instantaneous cross country speed while 
on the current task, \newline  compensated for altitude. Equivalent 
to instantaneous Pirker cross \newline country speed. \\

\hline

\verb|task_speed_hour| & Average cross country speed while on the 
current task \newline over the last hour, not compensated for altitude. \\

\end{tabularx}
\end{maxipage}

\begin{maxipage}
\begin{tabularx}{1.9\textwidth}{l|X}
Name & Description \\
\hline\hline

\verb|final_gr| & The required glide ratio over the ground to finish 
the task, given by \newline the distance to go divided by the height 
required to arrive at the safety \newline arrival height. \\

\hline

\verb|aat_time| & Assigned Area Task time remaining. \\

\hline

\verb|aat_time_delta| & Difference between estimated task time and 
AAT miminum time.\\

\hline

\verb|aat_distance| & Assigned Area Task distance around target points 
for remainder of \newline task. \\

\hline

\verb|aat_distance_max| & Assigned Area Task maximum distance possible 
for remainder of \newline task. \\

\hline

\verb|aat_distance_min| & Assigned Area Task minimum distance possible 
for remainder of  \newline task\\

\hline

\verb|aat_speed| & Assigned Area Task average speed achievable around 
target points \newline remaining in minimum AAT time. \\

\hline

\verb|aat_speed_max| & Assigned Area Task average speed achievable 
if flying maximum \newline possible distance remaining in minimum 
AAT time. \\

\hline

\verb|aat_speed_min| & Assigned Area Task average spped achievable 
if flying minimum \newline possible distance remaining in minimum 
AAT time.  \\

\hline

\verb|time_under_max_height| & The contiguous period the plane has 
been below the task \newline start max. height. \\

\hline

\verb|next_etevmg| & Estimated time required to reach next waypoint, 
assuming current \newline ground speed is maintained. \\

\hline

\verb|final_etevmg| & Estimated time required to complete task, 
assuming current ground \newline speed is maintained. \\

\hline

\verb|cruise_efficiency| & Efficiency of cruse, 1 indicates perfect 
MacCready performance \\


\end{tabularx}
\end{maxipage}

\subsection{Settings}

The Settings provides access to xcsoar settings, such as MC value.

The following attributes are provided by \verb|xcsoar.settings|:

\begin{maxipage}
\begin{tabularx}{1.9\textwidth}{l|X}
Name & Description \\
\hline\hline

\verb|mc| & The current set MacCready Value
$[{m/s}]$.\\

\hline

\verb|bugs| & The current used bug settings in terms of polar degradation.\\

\hline

\verb|wingload| & The current wingload.\\

\hline

\verb|ballast| & Ballast of the glider. 0 means no ballst, 0.3 means 30\% of
the maximum \newline  ballast the glider can carry.\\

\hline

\verb|qnh| & Area pressure for barometric altimeter calibration
$[{Pa}]$.\\

\hline

\verb|max_temp| & The forecast ground temperature
$[{K}]$.\\

\hline

\verb|safetymc| & The MacCready setting used, when safety MC is enabled 
for reach \newline calculations, in task abort mode and for determining arrival altitude 
at \newline airfields.\\

\hline

\verb|riskfactor| & The STF risk factor reduces the MacCready setting used to 
calculate \newline speed to fly as the glider gets low, in order to  compensate for risk.\\

\hline

\verb|polardegradation| & A permanent polar degradation, 1 means no degradation,
0.5 indicates the \newline glider's sink rate is doubled.\\

\hline

\verb|arrivalheight| & The height above terrain that the glider should arrive
at for a safe landing.\\

\hline

\verb|terrainheight| & The height above trerrain that the glider must clear during
final glide.\\

\hline

\verb|setmc(float value)| &  Sets the MacCready value\\

\hline

\verb|setbugs(float value)| & Sets the bugs, 1.0 means no bugs, 0.5 means 50\% polar
degradation.\\

\hline

\verb|setqnh(float value)| & Sets the QNH $[{Pa}]$\\

\hline

\verb|setballast(float value)| & Sets the ballst, 0 means no ballst, 0.5 means 50\%
of the maximum \newline ballst the glider can carry.\\

\hline

\verb|setmaxtemp(float value)| & Sets the maximum temperature $[{K}]$.\\

\hline

\end{tabularx}
\end{maxipage}

\subsection{Wind}

The Settings provides access to xcsoar wind data and settings.

The following attributes are provided by \verb|xcsoar.wind|:

\begin{maxipage}
\begin{tabularx}{1.9\textwidth}{l|X}
Name & Description \\
\hline\hline

\verb|wind_mode| & Wind mode, 0: Manual, 1: Circling, 2: ZigZag, 3: Both.\\

\hline

\verb|setwindmode(int value)| & Sets wind mode $[0-3]$.\\

\hline

\verb|tail_drift| & Determines whether the snail trail is drifted with the 
wind \newline when displayed in circling mode, 0: Off, 1: On. \\

\hline

\verb|settaildrift(bool value)| & Turns Taildrift Off / On $[{0,1}]$.\\

\hline

\verb|wind_source| & The Source of the current wind, 0: None, 1: Manual,
\newline 2: Circling, 3: Both, 4: External.\\

\hline

\verb|wind_speed| & The current wind speed 
$[{m/s}]$.\\

\hline

\verb|setwindspeed(float value)| & Sets manual the wind speed $[{m/s}]$.\\

\hline

\verb|wind_bearing| & The current wind bearing 
$[{degrees}]$.\\

\hline

\verb|setwindbearing(float value)| & Sets manual the wind bearing $[{degrees}]$.\\

\hline

\verb|clear()| & Clears the wind settings and calculations.\\

\hline

\end{tabularx}
\end{maxipage}

\subsection{Timers}

The class \verb|xcsoar.timer| implements a timer that calls a given
Lua function periodically.

\begin{lua}
xcsoar.timer.new(60, function(t)
  print("A minute has passed")
  t:cancel()
end)
\end{lua}

The following methods are available in \verb|xcsoar.timer|:

\begin{maxipage}
\begin{tabularx}{1.9\textwidth}{l|X}
Method & Description \\
\hline\hline

\verb|new(period, function)| & Create a new instance and schedule
it.  The period is a numeric value in seconds. \\

\hline

\verb|cancel()| & Cancel the timer. \\

\hline

\verb|schedule(period)| & Reschedule the timer. \\

\end{tabularx}
\end{maxipage}

\subsection{Legacy}

Before version 7.0, \xc was adapted using the \emph{InputEvent}
subsystem (see Appendix~\ref{cha:file_formats}).  During the Lua
transition, Lua scripts can invoke InputEvents, for example:

\begin{lua}
xcsoar.fire_legacy_event("Setup", "basic")
xcsoar.fire_legacy_event("Zoom", "basic")
\end{lua}

This function will be removed before the final 7.0 release.
