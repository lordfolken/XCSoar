This describes the setup of a development environment suitable to compile \xc for most supported platforms. The manual focuses on debian-based flavors of linux, but should be easily adapted to other distributions. It is developed with Ubuntu 16.04 LTS, and was last changed in May 2016 for \xc 7.0 preview  6.

\section{Install base system}\label{sec:developsetup-base}
The procedure uses the standard install, during/after which you should update all packages to the current level using

\begin{verbatim}
sudo apt-get update
sudo apt-get upgrade
\end{verbatim}

\section{Platform-independent components}\label{sec:developsetup-general}

Components required for all platforms, plus git for downloading the source and uploading commits.

\begin{verbatim}
sudo apt-get install make \
  librsvg2-bin xsltproc \
  imagemagick gettext \
  git quilt
\end{verbatim}
Some of these packages might be already present in your distribution (e.g. in the case of Ubuntu).

It is also recommended (though not required) to install \texttt{ccache}, which can speed up building a lot, when rebuilding after a \texttt{clean}.
\begin{verbatim}
sudo apt-get install ccache
\end{verbatim}


\section{Download source code}\label{sec:developsetup-sourcecode}

Download the source code of \xc by executing /textt{git} in the following way in your project directory
\begin{verbatim}
git clone git://git.xcsoar.org/xcsoar/master/xcsoar.git

# download submodule
cd xcsoar
git submodule init
git submodule update
\end{verbatim}

\section{For Unix builds}

Make sure your distribution contains a gcc compiler version 4.9 or higher in order to build the Unix target.

In addition, the following packages are needed:

\begin{verbatim}
sudo apt-get install make g++ \
  zlib1g-dev \
  libsdl1.2-dev libfreetype6-dev \
  libpng-dev libjpeg-dev \
  libtiff5-dev libgeotiff-dev\
  libcurl4-openssl-dev \
  liblua5.2-dev lua5.2-dev\
  libxml-parser-perl \
  librsvg2-bin xsltproc \
  imagemagick gettext \
  libegl1-mesa-dev\
  fonts-dejavu
\end{verbatim}

\section{For Android  builds}
\subsection{Install JDK}
First of all, we need Java to work with Android. We use OpenJDK 8 in this example, but OpenJDK 7 should also work.

\begin{verbatim}
sudo apt-get install openjdk-8-jdk ant
sudo apt-get install vorbis-tools automake
\end{verbatim}

\subsection{Install Android SDK and NDK}

The SDK and NDK are installed in the user directory. The \texttt{Makefile} expects them in specific directories, therefore we create some symlinks.

\begin{verbatim}
cd ~/
mkdir opt
cd opt/

# Download and extract SDK
wget "http://dl.google.com/android/android-sdk_r23-linux.tgz"

tar xvfz android-sdk_r23-linux.tgz
mv android-sdk-linux/ android-sdk-linux_x86/
ln -s android-sdk-linux/ android-sdk-linux_x86
rm android-sdk_r23-linux.tgz

# Download NDK installer

wget "http://dl.google.com/android/\
repository/android-ndk-r12b-linux-x86_64.zip"

unzip android-ndk-r12b-linux-x86_64.zip
\end{verbatim}

We need to add the paths by appending the following lines to \texttt{\textasciitilde/.profile} and \texttt{\textasciitilde/.bashrc} --- here relative paths work.

\begin{verbatim}
# Android SDK
export ANDROID_SDK_HOME="~/opt/android-sdk-linux_x86"
export PATH="$PATH:$ANDROID_SDK_HOME/tools"
export PATH="$PATH:$ANDROID_SDK_HOME/platform-tools"

# Android NDK
export ANDROID_NDK_HOME="~/opt/android-ndk-r11"
export PATH="$PATH:$ANDROID_NDK_HOME"
\end{verbatim}

The Android SDK (which contains some 32bit components) seems to have trouble running on some 64bit distributions, the following packages fix this:

\begin{verbatim}
sudo apt-get install lib32z1 lib32ncurses5
sudo apt-get install libgcc1:i386
\end{verbatim}

Now, you should start the Android manager with the \texttt{android} command, and install the required packages. Use the GUI to select and install SDK 22, and the (preselected) newest tools. Be warned, these will need many GB of space on your harddisk! (approximately 15 GB for typical defaults)

Finally, for the \texttt{adb} debugger interface to work via USB, most distributions require manual fixing of the \texttt{udev} rules. This can be done in the following way:

\begin{maxipage}
\begin{verbatim}
#Create etc/udev/rules.d/99-android.rules file

echo "#Acer" > 99-android.rules
echo "SUBSYSTEM=="usb", SYSFS{idVendor}=="0502", MODE="0666"" >> 99-android.rules
echo "#ASUS" >> 99-android.rules
echo "SUBSYSTEM=="usb", SYSFS{idVendor}=="0b05", MODE="0666"" >> 99-android.rules
echo "#Dell" >> 99-android.rules
echo "SUBSYSTEM=="usb", SYSFS{idVendor}=="413c", MODE="0666"" >> 99-android.rules
echo "#Foxconn" >> 99-android.rules
echo "SUBSYSTEM=="usb", SYSFS{idVendor}=="0489", MODE="0666"" >> 99-android.rules
echo "#Garmin-Asus" >> 99-android.rules
echo "SUBSYSTEM=="usb", SYSFS{idVendor}=="091E", MODE="0666"" >> 99-android.rules
echo "#Google" >> 99-android.rules
echo "SUBSYSTEM=="usb", SYSFS{idVendor}=="18d1", MODE="0666"" >> 99-android.rules
echo "#HTC" >> 99-android.rules
echo "SUBSYSTEM=="usb", SYSFS{idVendor}=="0bb4", MODE="0666"" >> 99-android.rules
echo "#Huawei" >> 99-android.rules
echo "SUBSYSTEM=="usb", SYSFS{idVendor}=="12d1", MODE="0666"" >> 99-android.rules
echo "#K-Touch" >> 99-android.rules
echo "SUBSYSTEM=="usb", SYSFS{idVendor}=="24e3", MODE="0666"" >> 99-android.rules
echo "#KT Tech" >> 99-android.rules
echo "SUBSYSTEM=="usb", SYSFS{idVendor}=="2116", MODE="0666"" >> 99-android.rules
echo "#Kyocera" >> 99-android.rules
echo "SUBSYSTEM=="usb", SYSFS{idVendor}=="0482", MODE="0666"" >> 99-android.rules
echo "#Lenevo" >> 99-android.rules
echo "SUBSYSTEM=="usb", SYSFS{idVendor}=="17EF", MODE="0666"" >> 99-android.rules
echo "#LG" >> 99-android.rules
echo "SUBSYSTEM=="usb", SYSFS{idVendor}=="1004", MODE="0666"" >> 99-android.rules
echo "#Motorola" >> 99-android.rules
echo "SUBSYSTEM=="usb", SYSFS{idVendor}=="22b8", MODE="0666"" >> 99-android.rules
echo "#NEC" >> 99-android.rules
echo "SUBSYSTEM=="usb", SYSFS{idVendor}=="0409", MODE="0666"" >> 99-android.rules
echo "#Nook" >> 99-android.rules
echo "SUBSYSTEM=="usb", SYSFS{idVendor}=="2080", MODE="0666"" >> 99-android.rules
echo "#Nvidia" >> 99-android.rules
echo "SUBSYSTEM=="usb", SYSFS{idVendor}=="0955", MODE="0666"" >> 99-android.rules
echo "#OTGV" >> 99-android.rules
echo "SUBSYSTEM=="usb", SYSFS{idVendor}=="2257", MODE="0666"" >> 99-android.rules
echo "#Pantech" >> 99-android.rules
echo "SUBSYSTEM=="usb", SYSFS{idVendor}=="10A9", MODE="0666"" >> 99-android.rules
echo "#Philips" >> 99-android.rules
echo "SUBSYSTEM=="usb", SYSFS{idVendor}=="0471", MODE="0666"" >> 99-android.rules
echo "#PMC-Sierra" >> 99-android.rules
echo "SUBSYSTEM=="usb", SYSFS{idVendor}=="04da", MODE="0666"" >> 99-android.rules
\end{verbatim}
\end{maxipage}
\begin{maxipage}
\begin{verbatim}
echo "#Qualcomm" >> 99-android.rules
echo "SUBSYSTEM=="usb", SYSFS{idVendor}=="05c6", MODE="0666"" >> 99-android.rules
echo "#SK Telesys" >> 99-android.rules
echo "SUBSYSTEM=="usb", SYSFS{idVendor}=="1f53", MODE="0666"" >> 99-android.rules
echo "#Samsung" >> 99-android.rules
echo "SUBSYSTEM=="usb", SYSFS{idVendor}=="04e8", MODE="0666"" >> 99-android.rules
echo "#Sharp" >> 99-android.rules
echo "SUBSYSTEM=="usb", SYSFS{idVendor}=="04dd", MODE="0666"" >> 99-android.rules
echo "#Sony Ericsson" >> 99-android.rules
echo "SUBSYSTEM=="usb", SYSFS{idVendor}=="0fce", MODE="0666"" >> 99-android.rules
echo "#Toshiba" >> 99-android.rules
echo "SUBSYSTEM=="usb", SYSFS{idVendor}=="0930", MODE="0666"" >> 99-android.rules
echo "#ZTE" >> 99-android.rules
echo "SUBSYSTEM=="usb", SYSFS{idVendor}=="19D2", MODE="0666"" >> 99-android.rules
sudo mv -f 99-android.rules /etc/udev/rules.d/
sudo chmod a+r /etc/udev/rules.d/99-android.rules
\end{verbatim}
\end{maxipage}

\section{Optional: Eclipse IDE}
One of the most widespread IDEs is eclipse. It is not limited to Android, and can be used for all targets. It is not required for \xc, but its installation is described here as an example. Eclipse is quite heavyweight, and many developers prefer other IDEs for \xc development.

To install, download the eclipse installer (Sometimes called "\emph{Ooomph!}" for some reason) from here:\\
\texttt{https://www.eclipse.org/downloads/}

Important: Install the CDT version of eclipse for C development, not the Android/Java package, even if you plan developing for Android. In addition, it is very convenient to install the git support (egit).

The current stable version is \emph{eclipse mars} (4.5) and works with OpenJDK 7 or 8, the new \emph{eclipse neon} 4.6, currently RC2, is also quite stable, and requires OpenJDK 8. Both can be installed with the installer.

You can also install the ADT (Android development tools) package for better integration with Android.

Next, create a new project, by generating a make project from existing sources files. Choose your xcsoar source directory which contains the makefile.

Important: After you have added the sources, eclipse will start indexing all files. If you have already started \texttt{make} before this time, then a lot of files have been downloaded for the various libraries which are exctracted/built within the \xc directory (most notably the boost libraries). Indexing all these takes a very long time, and a lot of heap space, so you should probably stop the indexer right away. In addition you should probably exclude these directories from the indexer for the future.

For this, in the C/C++ scope, right-click on the "output" directory in the file tree on the left side, select "Properties", then "Resource/Resource Filters" and add a filter. In the "add filter" dialog, choose "exclude all", "files and folders", "all children (recursive)" and set the Filter details to "Name matches *".
This will exclude the output tree from the indexer, leading to a minimal index.

\section{Build setup for the Manual}
To set up the LaTeX -- specifically, LuaLaTex -- system for compiling the manual, you need to install the following packages in addition to the steps from section \ref{sec:developsetup-base}, \ref{sec:developsetup-general} and \ref{sec:developsetup-sourcecode}:

\begin{verbatim}
sudo apt-get install texlive-luatex \
  texlive-latex-base texlive-latex-extra
\end{verbatim}

Now you can compile the manuals with 

\begin{verbatim}
make manual
\end{verbatim}

This creates a lot of warnings, but this is normal.

\section{Optional: modern LaTeX editor for editing the Manual}
Most people today edit LaTeX files in specific editors, as this is much more comfortable and efficient. This is highly recommended especially if you are not very familiar with LaTeX: learning it is very easy with a modern editor. Here, we install TeXstudio as an example, as it is very widespread and supports the rather rare LuaLaTeX well.

To install, get the relevant package:

\begin{verbatim}
sudo apt-get install texstudio
\end{verbatim}

As the directory tree of \xc is very unusual for a LaTeX project, we need to make some special configurations in order to allow for quick compiling from within the editor, and for full synctex functionality:

In "Options / Configure TeXStudio", enable "show advanced options".

In "Options / Configure TeXStudio / Commands / Commands / LuaLaTeX", replace
\begin{verbatim}
lualatex -synctex=1 -interaction=nonstopmode %.tex
\end{verbatim}
with
\begin{verbatim}
lualatex -synctex=1 -interaction=nonstopmode
 -output-directory=?a)../../../output/manual %.tex
\end{verbatim}

In "Options / Configure TexStudio / Build / Build Options / Addition Search Paths":\\
Enter in \emph{both} fields ("Log file" and in the field "PDF File"):
\begin{verbatim}
../../../output/manual/
\end{verbatim}


Add the following line to \emph{both} the \texttt{.profile} and the \texttt{.bashrc} file of your user directory:
\begin{maxipage}
\begin{verbatim}
export TEXINPUTS="..:../../../output/manual:../../../output/manual/en:../../..:"
\end{verbatim}
\end{maxipage}


Finally, you need to run "make manual" in the \xc base directory at least once from the command line before you can compile from within the TexStudio interface. This creates the path structure and generates the figure files which are included into the manual. Of course, if you change figures, you might have to run "make manual" again.

Inside TeXStudio, open the file "XCSoar-manual.tex" (or one of the other root files) and right-click on this file to "set as explicit root document", in the structure view on the left. Now you are good to go. Make changes and press F5 to see the result immediately.





