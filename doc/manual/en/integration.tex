\chapter{Integration}\label{cha:Integration}

\section{Building an XCSoar system}
\emph{Integration} is bringing together all things and actions in an entity for  
building a \emph{system} providing desired functionality. Things needed are 
hardware components, software, database files, configuration, power supply e.a. 
and glue components. As you might have expected, this chapter deals with 
integration of XCSoar, with the components of interest:
\begin{itemize} 
\item Hardware XCSoar runs on: Pocket PC, Android...
\item XCSoar software
\item Flight information databases: terrain, waypoint, weather, polar, 
flarmnet, airfield details...
\item Instruments: barometer, vario, Flarm, logger, horizon...
\item Backup components: second XCSoar unit
\item Mounting
\item Configuration: XCSoar internally and connections with
\item User training
\end{itemize}
No system ever runs without using some components providing no primary function 
of interest, but some ``glue'', holding things together.  Most likely your XCSoar
system needs one ore more of these glue components as are:
Cabling as well as data, signal, and power \emph{converter} and signal 
\emph{multiplexer}.

Finally, there are things, a pilot may not think of when doing integration work. 
However, if you want to get the max from your XCSoar system, allow for optimal 
interaction with other systems via \emph{interfaces}.
\begin{itemize}
\item Remote tracking server (skylines) / Cellular network reception and coverage
\item GPS satellite network / GPS reception
\item Remote database servers / data translator
\item XCSoar.org / contact channels as given in chapter introduction
\end{itemize}
So far, it is time to end with systems theory, now dive into reality. With 
countless valuable system setups possible, just a few examples of XCSoar systems 
are given. This is meant as a starting point for the 
keen pilot. In no way this manual represents a full-fledged integration guide.  
The five examples described will follow the path of integrating more components, 
starting with a ``basic'' system, consisting of one single piece of hardware only.

\section{Exemplary XCSoar system setups}
Due to the fact, this manual is not intended as an advertising publication, 
several component's terms used in the following stand for a class of devices. For 
gathering information on hardware products, please consult XCSoar's website 
\url{http://www.xcsoar.org/hardware/}, XCSoar's forum \url{http://
forum.xcsoar.org/} or ask google. With terms \emph{emphasized} you are already 
equipped with keywords for invoking a search and discover tour.

\subsection*{XCSoar Basic}
\subsubsection*{Setup} Handheld with XCSoar and a built in GPS receiver and basic 
data stock, provided by XCSoar website and the national agency of your country, 
providing airspace data.

\subsubsection*{Functionality:} An XCSoar basic setup will give you an almost 
perfect moving map with a great basic flight information system (FIS) 
functionality. The glide computer will give you flight-related tactical 
\emph{and} statistical data if polar configuration is correct. Please note, that 
realtime computations as is actual climb or sink rate may suffer drastically from 
poor GPS reception, as in built GPS receivers often are cheap add-ons in a mobile 
device's budget. In any case, please give it a try in the air.  Some of these 
cheap GPS receiver add-ons come with bad GPS antennas and sometimes are heavily 
dependent on local information of your cellular network provider. This dual mode 
locating function is called aGPS - assisted GPS. As mobile network coverage gets 
worse up in the air, your devices location fix follows. If your mobile device 
shows stable reception, you achieved a good one.

\subsubsection*{Application} XCSoar basic is the utmost club-friendly setup out 
of all.  With no additional hardware needed, you own a valuable system with no 
intrusion into club gliders necessary. A basic system might be helpful if all of 
your club mates are rather conventionally oriented or if you are a parachutist 
with very little instrument space.

\subsection*{XCSoar Classic}
\subsubsection*{Setup} An XCSoar classic system is a basic system plus a 
\emph{Flarm} connected. As long as the family of Flarm and Flarm like hardware 
uses serial communication ports, a first piece of glue hardware appears necessary 
in many cases. There are several pieces of hardware available, converting signals 
form Flarm's serial port to either USB or even radio link Bluetooth.

\subsubsection*{Functionality} As you might have expected, Flarm itself gives you 
a collision detection system. Added integration value first of all might be a 
rock-steady GPS reception. With precise GPS fixes, XCSoar's actual readings 
improve a lot. Still, reception is heavily dependent on the pilot's mounting of 
the GPS antenna. Place it in a definite professional manner and verify reception.

Further added value is about Flarm devices, giving you an air pressure 
measurement. This enables XCSoar to compute QNH height. It is much more precise 
than height derived from the GPS calculations due to the fact, GPS height is the 
value with the worst precision out of all publicly available GPS measures. 
Uncertainty in height is around 50 meters.

Additionally, XCsoar takes the position data of others around you from Flarm and 
draws ``Flarm targets'' on your moving map. Displaying even climb rates of the
others, a classic XCSoar system aids in team flying. As long as reception is 
good, you do not have to call your mates whether their efforts in thermals are 
worth it or just for asking where they are actually. Last but not least, Flarm 
will bring in an IGC-approved logger. This you will need to participate in 
several contests, as the logger has to be approved and sealed.

\subsubsection*{Application} An XCSoar classic system also is club-glider 
friendly. If your club-gliders are equipped with a plug, intended for managing or 
downloading from Flarm devices, you can connect with almost no intrusion in the 
glider's instrumentation structure. Whatever you are willing to integrate in your 
XCSoar system, said functionalities of a classic system turn out to own a very 
good value / integration effort ratio.

\subsection*{XCSoar Classic+}
\subsubsection*{Setup} Bring in an additional \emph{ADS-B} receiver. With rapid 
market penetration of Flarm, a very welcome collision detection coverage in the 
glieder scene is already achieved. However, this is not the case with motor 
driven vehicles. If you want to have motorized targets to be depicted on your 
moving map, consider an ADS-B receiver on your integration list. Integrating a 
Flarm and an ADS-B receiver is a first example for integrating two sources of 
data. Possibly you need a multiplexer in case your hardware XCSoar runs on, has 
only one communication port. Either purchase a hardware mutliplexer or use 
Bluetooth functionality. Most smartphones come with a Bluetooth built in 
multiplex functionality. Another way is to achieve a \emph{Power Flarm} device, 
already equipped with a multiplexer for both Flarm and ADS-B data. 

\subsubsection*{Functionality} ADS-B stands for Automatic Dependent Surveillance 
- Broadcast. 
This radar based system broadcasts position of equipped aircrafts even if they 
are not pinged by ground primary radar. If you plan to use such a device, 
please plan to do a careful system setup too. As long as commercial aircrafts 
travelling in upper flight levels, you are not interested in, you probably need
no information about on your map. Also, radar transponders are made as long 
range systems. Simply set up several filterings in your system in order not to 
crowd your moving map.

\subsubsection*{Application} Increased situation awareness support. However, many 
pilots judge ADS-B being information overkill as do many representatives of 
ground control organisations. \emph{Please} take your time when reading 
discussions on ADS-B usage in gliders \emph{before} purchasing hardware. 
Technically speaking ADS-B is brilliant as is Flarm.

\subsection*{XCSoar Tactic}
\subsubsection*{Setup} Integrate an electronic variometer with your classic 
System, an \emph{eVario}. Since an eVario is expected to be a second piece of 
hardware to be connected, you might need a data multiplexer for connecting both, 
Flarm and eVario to your XCSoar hosting hardware. Take your time studying the 
hardware features of your eVario of interest. There are some devices selling, 
that bring with them a multiplexer for merging GPS and other data from Flarm with 
their own measurements presented on an output. Additionally a voltage converting 
power supply might be included on that output port, ready for supplying a ``data
plus power package'' for your XCSoar hardware.

\subsubsection*{Functionality} XCSoar's tactical computations will give you 
valuable estimates of positions in reach, provide heading, bearing and 
sophisticated information on times needed to reach waypoints and much more... 
With the actual wind vector being a very important value, those computations rely 
on, an eVario will greatly improve XCSoar's wind estimates by turning the ``value''
wind estimate to a rather actual wind ``measurement'' kind value (still the wind 
vector remains an estimate). Obviously, an eVario also will give you precision 
actual lift/sink measurements.

With all tactical computations being dependent on your actual estimates of 
MacCready and polar degradation, you might find synchronising your input to both, 
eVario and XCSoar useful. It is just a matter of setting up things to do so.

\subsubsection*{Application} Get the best tactical data in flight, a glider pilot 
can get. :-))

\subsection*{XCSoar Competition}
\subsubsection*{Setup} Put in additional data files in your Tactic system as are 
tasks and waypoint databases amongst others. To input varous file formats it 
might be necessary to involve a data converter. Additionally you might set up 
extra screen pages, covering competition related measures.

In case, a competition hosting / umbrella organisation alters task rules, there 
is a chance, something turns out to be done by XCSoar's developers. At the latest 
by facing a newly introduced crude task rule, you develop awareness of 
developer's support urgently needed. Exactly then the XCSoar organisation is 
perceived as being part of your XCSoar system.

Please input any \emph{important} information to the XCSoar community promptly. 
All the developers will do their best keeping up with changes as are competition 
rules for example. But please keep in mind, XCSoar is an open source project, 
performed by \emph{volunteers}. The earlier you provide information on changes, 
the better your chance is gaining support by the XCSoar community. Nevertheless, 
all said holds true for commercial suppliers as well.  

\subsubsection*{Functionality} The task files put in will ensure, you are 
equipped with precise task data. No error-prone manual input.

The additional screen pages will give you InfoBox readings, related to several 
phases of the competition. Just to give you a clue, the InfoBox ``Start open /
close countdown'' appears to be useless in all day flying situations? Put it on a
screen page other than your standard ensemble.

\subsubsection*{Application}
No comment.

\subsection*{XCSoar Reference / Developer}
\subsubsection*{Setup} Put in either component, not yet mentioned in your XCSoar 
setup. Please be aware, that the XCSoar project is targeting the best support of 
VFR type soaring possible. With inputting other functionality, you probably enter 
a space of ongoing experiments having attracted some attention in the XCSoar 
community. Especially when integrating an \emph{AHRS} system: You are handling 
non-certified components. \tip \emph{ALL THINGS, JUST EVERYTHING} you do with 
XCSoar software is at your own risk. With focusing on setups up to the Tactic 
class of systems, the XCSoar community pays lot of attention on providing stable 
code (software). With some integration issues being rather exotic, you leave the 
depth of field of primary focus. There might be much too less users for providing 
a secure testing base for exotic features. Although, all the times XCSoar's code 
itself is maintained for stability as long as you are involving \emph{released} 
code.

AHRS stands for Attitude Heading Reference System.

\subsubsection*{Functionality} Although not being in focus of every XCSoar user, 
you should follow the development process in order to get an up to date survey on 
additional functionality. There is a chance it might pay off. Just to give a few 
examples:
\begin{itemize}
\item With an AHRS type device you are able to use an artificial horizon.
\item RASP data are for lee wave pilots
\item Additional airspace files, focusing on \emph{glider} pilot's needs.
\item Hardware command input devices, provided by enthusiastic hardware 
developers (there even were few cases, pilots involved gamepad hardware!)
\item About possible future things: New components classes to be integrated  
under discussion as might be transceivers, motor management, transponders etc. 
(as is with January, 2014)
\end{itemize}

\subsubsection*{Application} Whenever you think of a new killer feature to be 
integrated or just want it all, set up a Reference / Developer XCSoar system.
Additionally, feel deeply encouraged to test software previews or alpha/beta 
versions after having gained experience with XCSoar's behavior. Help improve 
stability of software updates / upgrades, or provide some new code. In other 
words:

\textsl{Think of the XCSoar project as being part of your passion for gliding and 
contribute. Please do not only follow this passion by soaring. Get involved in 
XCSoar's development process and discover the power and beauty of the open-source 
credo. 
Doing so you will discover a close congeniality. A Reference / Development system 
is a worthy tool to become a member of the project.}
