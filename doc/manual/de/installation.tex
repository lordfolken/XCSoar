\chapter{Installation}\label{cha:installation}

Um {\textsf  XCSoar} zum Laufen zu bringen benötigst Du folgendes:


\begin{itemize}
\item ein Gerät, auf dem {\textsf  XCSoar} läuft.
\item {\textsf  XCSoar} an sich
\item einen GPS Empfänger
\item ein Wegpunkt-File
\item ein Luftraum-File
\item ein Karten-File (Kartendatenbank bzw. Gelände)
\end{itemize}

\section{Kompatibilität}

\subsection*{Geräte auf denen  {\textsf  XCSoar} läuft}

{\textsf  XCSoar} läuft derzeit auf folgenden Plattformen:

\begin{itemize}
\item Handys und Tablets mit Android 1.6 oder neuer\\
  Beispiel: Dell Streak, Samsung Galaxy S II, HTC Desire HD, Samsung Tablet 10.1N
  Motorola Xoom
\item PDAs mit Pocket PC 2000, 2002, 2003 \\
  Beispiel: Asus A600, Compaq iPaq 3800, 38xx, iPaq 3900, 39xx
\item PDAs mit Windows Mobile \\
  Beispiel: iPaq hx4700, Dell Axim x51v
\item PNAs mit  Windows CE 3.0 oder neuer \\
  Beispiel: HP314, Mio400
\item Triadis Altair
\item LX MiniMap
\item Windows 2000 oder neuer
\item Linux
\item Mac OS X
\end{itemize}

\subsection*{GPS, Logger, Varios}

{\textsf  XCSoar}  ist kompatibel zu allen Geräten, welche GPS-Daten im NMEA-Format ausgeben.

Die meisten modernen Android Geräte haben einen eingebauten GPS-Empfänger, der diese Daten zur Verwendung 
ausgibt. Mitunter jedoch ist es ratsam, diese Daten aus einem anderen Gerät zu verwenden, wenn man z..B.\ Luftverkehrsdaten aus einem 
\fl benutzen und anzeigen möchte oder aber noch weitere Features -wie nachfolgend aufgelistet- nutzen möchte:



\begin{itemize}
\item Geschwindigkeitssensor (TAS) erlaubt schnelle und eine exakte Ermittlung des Winds ohne Kurbeln zu müssen
\item ein extern angeschlossenes Vario wird die Zentrierhilfe unterstützen und erheblich beschleunigen
\item Aufgaben können deklariert und ausgetauscht werden (laden und speichern z.B. in IGC Logger ),   Flüge können ausgelesen werden 
\item manche Varios erlauben eine Synchronisation z.B.\ des MC-Wertes und anderen Werte mit {\textsf  XCSoar} 
\end{itemize}

\subsection*{Unterstützte externe Geräte und deren Funktionen}\label{sec:supported-varios}

\newcommand{\y}[0]{{ $\surd$ }}
%{0.8\textwidth}
\noindent\makebox[\textwidth]{%
\begin{tabular}{l|ccc|cc|cc|c}
       \multicolumn{1}{r}{\bf Unterstütze:} & \multicolumn{3}{c|}{\bf -Funktionen} & \multicolumn{5}{c}{\bf -Daten} \\
NMEA Gerät & 
  \begin{sideways} Deklaration\end{sideways} & 
  \begin{sideways} Fernbedienung\end{sideways} & 
  \begin{sideways} Download\end{sideways} &
  \begin{sideways} TAS \end{sideways} & 
  \begin{sideways} Vario\end{sideways} & 
  \begin{sideways} Baro. Höhe\end{sideways} & 
  \begin{sideways} Wind\end{sideways} &
  \begin{sideways} G-Sensor\end{sideways} \\
\hline
%                    _Decl_Remo_Down_Airs_Vari_Baro_Wind_Gsen_
Borgelt B50          &    & \y &    & \y & \y & \y &    &    \\
CAI 302              & \y & \y & \y & \y & \y & \y & \y & \y \\
CAI GPS Nav          &    &    &    &    &    &    &    &    \\
Condor               &    &    &    & \y & \y & \y & \y &    \\
\hline
Digifly Leonardo     &    &    &    & \y & \y & \y & \y &    \\
EW Logger            & \y &    &    &    &    & \y &    &    \\
EW microRecorder     & \y &    &    &    &    & \y &    &    \\
FLARM                & \y &   & \y  &    &    & \y &    &    \\
\hline
%                    _Decl_Remo_Down_Airs_Vari_Baro_Wind_Gsen_
Flymaster F1         &    &    &    &    & \y & \y &    &    \\
Flytec 5030          &    &    &    & \y & \y &    &    &    \\
GTAltimeter          &    &    &    &    &(\y)& \y &    &    \\
ILEC SN10            &    &    &    &    & \y & \y & \y &    \\
\hline
IMI ERIXX            & \y &    & \y &    &    &    &    &    \\
LX20, Colibri        & \y &    & \y &    &    & \y &    &    \\
LXNAV Nano           & \y &    & \y &    &    &    &    &    \\
\hline
%                    _Decl_Remo_Down_Airs_Vari_Baro_Wind_Gsen_
LXNAV V7             &    & \y &    & \y & \y &    &    &    \\
PosiGraph            & \y &    &    &    &    & \y &    &    \\
Triadis Altair (pro) & \y &    &    &    &    & \y &    &    \\
Triadis Vega         &    & \y &    & \y & \y & \y &    & \y \\
\hline
Vaulter              &    & \y &    & \y & \y & \y & \y & \y \\
Volkslogger          & \y &    & \y &    &    & \y &    &    \\
Westerboer VW1150    &    & \y &    & \y & \y & \y &    &    \\
Westerboer VW921, 922
                     &    & \y &    & \y & \y & \y &    &    \\
\hline
Zander / SDI         &    & \y &    & \y & \y & \y & \y &    \\

\end{tabular}}


Während die meisten CE-basierten Geräte über eine serielle Schnittstelle verfügen, 
ist diese Art der Schnittstelle bei den meisten, modernen Android-Geräten nicht mehr vorhanden. 
Diese können entweder Bluetooth oder aber das Android IOIO-Board verwenden. 


Um Bluetooth zu benutzen, verbinde das Gerät mit einem Bluetooth $\leftrightarrow$ Seriell-Adapter, wie z.B.\ 
 dem K6-Bt oder dem Glidertools VFBT-1.


\section{Software Installation der Software}\index{Installation}

Die Software ist auf der {\textsf  XCSoar}-Homepage frei herunterzuladen. 
In diesem Kapitel wird behandelt, {\sl wie} die Software heruntergeladen und installiert wird.

\subsection*{Betriebssystem Android}\index{Installation!unter Android}

Suche nach {\textsf  XCSoar} in Googles Android-Marktplatz (PlayGround), oder installiere das \verb|apk|-
File manuell.  Kopiere diese Datei auf eine SD-Karte in das Verzeichnis \verb|XCSoarData|.

\subsection*{Auf einem PDA (Windows Mobile, PocketPC)}\index{Installation!unter Pocket PC, Windows Mobile}

Wähle eines dieser Files:

\begin{description}
\item[\texttt{PPC2000}] Pocket PC 2000/2002, Windows CE 3.0
\item[\texttt{PPC2003}] Pocket PC 2003, Windows CE 4.0
\item[\texttt{WM5}] Windows Mobile 5 oder neuer
\item[\texttt{WM5X}]Windows Mobile 5 oder neuer mit XScale CPU oder besser(z.B. hx4700)
\end{description}

\sketch{figures/XCS_Today.png}
Lade das File \verb|XCSoar.exe| herunter und kopiere es auf eine SD Karte.  
Starte das File z.B.\ über den Explorer. 

Ein andere Methode ist, {\textsf  XCSoar} auf PDA über das CAB-File zu installieren.  Lade dies File herunter auf Deine SD-Karte, 
anschließend  doppelklicke darauf und {\textsf  XCSoar} wird auf Deinem PDA installiert werden. 

Nach der Installation werden das {\textsf  XCSoar}-Symbol sowie die beiden Schriftzüge ''FLY'' und ''SIM'' aufgelistet
dem ''Heute''-Schirm sichtbar sein. (s.\ Bild) 


\subsection*{Auf einem PNA (Windows CE)}\index{Installation!unter Windows CE}

Lade das File \verb|XCSoar.exe| (Ziel wie oben ''WM5'') auf eine SD Karte herunter. 
Führe es z.B.\ mit dem Explorer aus.

\subsection*{Auf einem Windows PC}\index{Installation!unter Windows}
  
Lade das entsprechende File \verb|XCSoar.exe| (Ziel ''PC'') auf Deinen Rechner (Platte) herunter und führe es aus.


\subsection*{Unter Unix/Linux}\index{Installation!unter Linux}

Das herunterzuladende File ist \verb|xcsoar_XXX.deb|, wobei \verb|XXX| die Version und Plattform darstellt.  
z.B.\ \verb|xcsoar_6.0.4_i386.deb|. Dies ist ein Debian Paket und für 32bit Betriebssysteme mit Intel-Prozessoren gedacht.
Es kann installiert werden mit:
\begin{center}
\verb|sudo dpkg -i xcsoar_XXX.deb|.
\end{center}
Um zu sehen, wo die ausführbaren Dateien und die anderen Dateien installiert wurden, führe \verb|dpkg-query -L xcsoar| 
aus. 

Alle weiteren Dateien (Karten, Lufträume etc.\ müssen in \verb|~/.xcsoar/XCSoarData/| kopiert werden.
Wenn das Verzeichnis \verb|~/.xcsoar| nicht existiert, wird es beim ersten Aufruf von {\textsf  XCSoar} angelegt.


\section{Daten Files}\label{InstallationDatenfiles}\index{Installation!Datenfiles}

Um alle Funktionen von {\textsf  XCSoar} nutzen zu können, werden weitere Daten und Files benötigt.
Hierzu zählen Lufträume, Geländedaten, Topologiedaten eine Polare, Wegpunkte und so weiter. 


Die Nutzung dieser Daten und Files wird in Kap.~\ref{cha:data-files} ausführlich beschrieben. 

Alle Files sollten unbedingt in das Verzeichnis \texttt{XCSoarData} kopiert bzw. \ gespeichert werden. 
Dies Verzeichnis muß  in einem bekannten Verzeichnis befindlich sein, damit {\textsf  XCSoar} bekannt ist, wo es nach entsprechenden 
Daten-File suchen muß: 

 
\begin{description}
\item[Windows PC]
\texttt{XCSoarData} befindet sich im Persönlichen Verzeichnis (''\texttt{Eigene Dateien}'')
\item[Windows Mobile PDA/PNA]
- Wenn ein Verzeichnis \texttt{XCSoarData} im gleichen Verzeichnis wie {\textsf  XCSoar} existiert, dann wird dies benutzt.  


- \texttt{XCSoarData} befindet sich auf der SD-Karte.  Wenn keine SD-Karte vorhanden ist, dann schaut {\textsf  XCSoar} in \texttt{Eigene Dateien} nach diesen Files

\item[Unix/Linux]
Das Verzeichnis heißt \verb|.xcsoar| und befindet sich im \verb|home| - Verzeichnis des Users. 
\item[Android Geräte]
\texttt{XCSoarData} befindet sich auf der SD-Karte.
\item[Altair]
Wenn {\textsf  XCSoar} sich auf einem USB-gerät (Stick) befindet, dann wir dies benutzt, andernfalls wird der interne Speicher benutzt. 
\end{description}


{\textsf  XCSoar} wird während der Laufzeit einige zusätzliche Files, wie z.B.\ eine Datei mit sämtlichen Konfigurationsdetails erzeugen.
Diese werden ebenfalls im \texttt{XCSoarData} Verzeichnis (Windows PC und Windows Mobile Geräte), oder aber im \texttt{.xcsoar} 
Verzeichnis(Unix/LinuxPC) abgelegt. 

Beim ersten Start wird {\textsf  XCSoar}  folgende Files erzeugen und anlegen: 
\texttt{Default.tsk} (Standard Aufgabe),  \texttt{default.prf} (Konfigurationseinstellungen), \texttt{xcsoar.log} 
(Ein Log-Schrieb des Startprozesses von {\textsf  XCSoar}\dots), 



Weiterhin werden drei Verzeichnisse angelegt: \texttt{cache}, \texttt{config} und \texttt{logs}.  

Während des Laufes von {\textsf  XCSoar} können (und werden) weitere Files angelegt wie z.B.\  Aufgaben-Files (\texttt{*.tsk}) 
und Flugaufzeichnungen (evtl.\ (\texttt{*.igc})


\section{Betrieb von {\textsf  XCSoar}}

{\textsf  XCSoar} wird kann in zwei Modi ausgeführt werden:
\begin{description}
\item[FLY] Dies ist der ''echte'' Flugmodus. Alle Ein - und Ausgänge betr.\ Signalaustausch sind aktiv
\item[SIM] Dies ist der Simulator-Modus. Keine Kommunikation mit externen Geräten per serieller Schnittstelle möglich. 
\end{description}

\subsection*{\al - Version}
Wenn der \al angeschaltet wird, fährt {\textsf  XCSoar} automatisch hoch (es ist quasi das das ''Betriebssystem'' des \al) 
Der PWR/ESC Knopf hat mehrer Funktionen:
\begin{description}
\item[Anschalten]  Drücken und halten für mehr als eine Sekunde: Die LED beginnt zu leuchten und {\textsf  XCSoar} wird starten, nachdem der \al gebootet hat.
\item[Ausschalten]  Drücken und Halten für mehr als drei Sekunden: Schaltet den Altair und somit {\textsf  XCSoar} aus.
\item[Escape] Kurzes Drücken des Knopfes wirkt wie die ESC-Taste eines PC: Wird benutzt als Abbruch-Funktion oder aber zum 
              Ausschalten von von Dialogen benutzt.
\end{description}

\achtung Die \al-Version von {\textsf  XCSoar} hat keine Simulator-Funktion

\subsection*{{\textsf  XCSoar}-PC Version}
Ein Doppelklick im entsprechenden Verzeichnis {\textsf  XCSoar} öffnet das Programm. 
{\textsf  XCSoar} kann auch mit Kommandozeilenoptionen oder über ein Batch-File mit Parametern geöffnet werden, 
die hier angegeben werden: 

\begin{description}
\item[-portrait] Anzeige ist 480 Pixels breit, 640 Pixels hoch
\item[-square] Die Anzeige ist 480 Pixel breit, 480 Pixel hoch
\item[-landscape] die Anzeige ist 640 Pixel breit, 480 Pixel hoch. Dies ist die Standardeinstellung.
Wenn Du nichts anderes angibst, wird diese Einstellung auf dem PC automatisch gewählt.
\item[-small] Dies stellt den Bildschirm in halber Größe dar. sinnvoll z.B.\ wenn {\textsf  XCSoar} in Verbindung mit dem Segelflugsimulator 
{\sc Condor} für das Trainig (oder gegen Langeweile) im Winter gestartet wird.
\end{description}

Um die Bildschirmdarstellung zu ändern ist es sinnvoll, sich in Windows eine Verknüpfung herzustellen. 
In den ''Eigenschaften'' sollte dann unter ''Ziel'' der Programm zusammen mit den Kommandozeilenparametern 
angegeben werden. (z.B.\ \texttt{xcsoar.exe -landscape}


\subsection*{{\textsf  XCSoar} Unix/Linux PC-Version}
Starte \verb|xcsoar| von der Kommandozeile oder aber erzeuge einen Link auf dem Desktop. 

Der Ort, an dem sich die ausführbare Datei befindet,  kann mit \verb|which xcsoar| herausgefunden werden.
Nicht alle Optionen wie bei der PC- Version sind hierbei nun möglich.
 
\subsection*{Nachladen von Daten-Files}
Das erste Mal, wenn {\textsf  XCSoar} gestartet wird, werden {\sl noch nicht} automatisch die Files nachgeladen, welche im 
angegebenen \verb|XCSoarData| abgespeichert wurden. Die kann erst geschehen, wenn {\textsf  XCSoar} das Konfigurationsfeile erstellt hat - also nach dem ersten Start.


Um {\textsf  XCSoar} kenntlich zu machen, wo sich die entsprechenden Files befinden, mache einen kurzen Doppelklick irgendwo auf die nackte Bildschirmfläche 
\sketch{figures/config-basic.png} nach dem Start irgendwo in der nähe des stilistisch abgebildeten Segelflugzeuges in der Mitte, klicke zweimal auf die Schaltfläche ''Konfig'' 
und wähle dann 
\begin{center}
\bmenut{Konfig.}{1/3}~\blink~\bmenut{Konfig.}{2/3}~\blink~\bmenut{System}{Einstellung} 
\end{center}
\achtung {\sl Im folgenden werden die Anzahl der Klicks auf die Menüs wie oben beschrieben nur noch mit {\bf einem} Kästchen dargestellt, d.h.\bmenut{Konfig.}{3/3} steht dann für {\bf 3-maliges Klicken} auf den Konfig.-Knopf
und damit für die dritte Seite des Konfigurations-Menüs}

Auf dieser allerersten Konfigurationseite kannst Du nun unter dem Punkt ''Standortdatei''  die Kartendatenbank, ein Luftraum- sowie ein Wegpunkt-File angeben.
Damit ist {\textsf  XCSoar} bis auf weitere sofort einsetzbar.
viele (wirklich viele\dots) andere Details können anschließend unter \bmenut{System}{Einstellung} an Deine Vorlieben und Bedürfnisse angepaßt werden. 
Die Konfiguration  ist aujsführlich beschrieben in Kap.\ref{cha:configuration} und unbedingt lesenswert. 

Nachdem diese Files hier einmal eingestellt wurden, werde sie bei jedem Start (es sei denn, Änderungen und/oder Neuinstallationen werden vorganommen)
automatisch wieder geladen.

\subsection*{Start-Up und Benutzer Profile}\index{Benutzerprofile}
Beim Start von {\textsf  XCSoar} wird automatisch nach einem Benutzerprofil gesucht. Wenn mehrere vorhanden sind, wird in einem Fenster ganz unten am rande nachgefragt, welcghes file benutzt werden soll.
Aus diesem Grunde macht (wirklich) es Sinn, den einzelnen Benutzerprofilen eindringliche, wiedererkennbare Namen zu geben, 
um diese schnell und eindeutig unterscheiden zu können. 

Wenn kein Profil explizit ausgewählt und geladen wird, wird automatisch das Profile des letzten Startes benutzt. 

Profile können sehr sinnvoll sein z.B. für  

\begin{itemize}
\item verschiedene Piloten
\item eines für Wettbewerbe, eines für Lustflug
\item Fliegen an unterschiedlichen Plätzen (zu Hause, Fliegerlager, Namibia\dots )
\end{itemize}


\subsection*{Simulator Modus (SIM)}
Der Simulatormodus ist entwickelt worden, um mit {\textsf  XCSoar} firm zu werden und bereits vor dem ersten Einsatz im Flugzeug 
Erfahrung mit der Bedienung sammeln zu können. Das unterscheidet {\textsf  XCSoar} von manch anderem fest eingebautem, erheblich teurerem 
 Gerät. 

Im Simulatormodus wird das auf dem Schirm sichtbare Flugzeug ganz einfach gesteuert:
Auf dem TouchScreen (oder mit der Maus), einfach mit dem Finger auf das Segelflugzeugsymbol tippen und in die Richtung 
streifen, wo es hingehen soll. Je länger man zieht/streift, desto schneller wird sich das Flugzeug bewegen. 

In der PC-Version und bei Geräten mit mehreren Knöpfen kann die Geschwidingkeit, Höhe und Richtung mit bestimmten Tasten geändert werdden: 

\achtung {\sl Diese Art der Bedienung ist naturgemäß nicht für TouchScreen-Geräte verfügbar!}


Die Höhe des Flugzeuges kann geändert werden mit einem Druck auf die GPS-Höhe in der \infobox{H GPS}-InfoBox, hier Hoch- und Runter-Tasten benutzen. 

Die Geschwindigkeit des Flugzeuges kann angepasst werden über die Übergrundgeschwindigkeit in der \infobox{V Gnd}-Infobox), hier die Hoch- oder Runter-Taste drücken (oder gedrückt lassen)

Der Kurs des Flugzeuges kann über Kurs-\infobox{Track}-InfoBox, angepasst werden, Hoch- oder Runter -Taste drücken oder halten. 

Wenn entweder\index{Simulator!Steuerung} \infobox{H GPS} oder\infobox{V ü Gnd}) ausgewählt ist, 
kann die Richtung des Flugzeuges auch über die Links- und Rechts-Tasten gesteuert werden.

Alle andere Funktionen, Knöpfe(Buttons) und Menüs arbeiten genau gleich wie im ''FLY'' Modus. 


\subsection*{Start Seite - Ladevorgang}\index{Geduld}
Während des Ladevorganges zeigt {\textsf  XCSoar} einen Ladebalken am unteren Rande des Fensters, 
welche die aktuell geladenen Files und Optionen darstellt. Kann als ''Ladevorgangsbalken'' angesehen werden;  auf PDA $>$ 10Jahre mit dataillierter Kartendatenbank mitunter 
relativ langsam (Geduld-bis zu 90sec können vergehen!!), auf aktuellen Geräten (PC, Android) vergehen meist keine 2 Sekunden (nur beim ersten mal..) 

Diese Seite zeigt auch die Version der Software.

\subsection*{Schließen des Programmes}
Bei PDA und PC-Version, kann {\textsf  XCSoar} vom Menü aus beendet werden. Hierzu Doppelklick auf die Kartenfläche und anschließend 
\begin{quote}
\bmenu{Beenden}
\end{quote}

Beim PC kann auch das ''normale'' Windows-Kreuz genutzt werden.
Der \al wird heruntergefahren und damit {\textsf  XCSoar} geschlossen durch Drücken auf den PWR-Knopf für mehr als zwei Sekunden.
